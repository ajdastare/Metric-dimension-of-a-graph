\documentclass[a4paper]{report}
\usepackage[slovene]{babel}
\usepackage[utf8]{inputenc}
\usepackage[T1]{fontenc}
\usepackage{bbm}
\usepackage{amsmath}
\usepackage{blindtext}
\title{Metrična dimenzija grafa }
\author{Nastja Košir, Ajda Stare}
\date{\today}
\begin{document}
	\maketitle

\section*{Navodilo}
For an ordered subset $W = (w_1 , \dots, w_k ) $ of vertices of a graph G, the distance vector of a vertex v $\in$ V(G)  to  W  is  $(d(v, w_1 ), d(v, w_2 ), \dots, d(v, w_k ))$ . The metric dimension of a graph G is the size of the smallest set of vertices W  $\subseteq $ V(G) such that every two different vertices of the graph have distinct distance vectors. Solve the metric dimension problem as ILP. Its LP relaxation is known as the fractional metric dimension of a graph. Experimentally determine how much these two invariants can differ. Also, try to determine them for trees or any other class of your interest. How far you can go with determining the metric dimension of hypercubes, or some other cartesian product graphs?\\

\section*{Opis problema}
Naj bo $G=(V,E)$ neusmerjen, povezan graf brez zank in večkratnih vozlišč. Za urejeno podmnožico $ W =(w_1 , \dots, w_k )$ vozlišč grafa G, definiramo\\$r(v|W) = (d(v,w_1),\dots, d(v,w_k))$, kot  metrično predstavitev vozlišča $v$ , glede na množico W. Množico W imenujemo \textbf{rešljiva (razrešitvena) množica} grafa G, če za poljubni  različni vozlišči $u,v$ grafa G velja $r(u|W)\neq r(v|W)$ (če bi veljalo $r(u|W)= r(v|W)$ bi pomenilo $u = v$ ). Torej, če je W rešljiva množica moči k, za graf G reda n, potem množica $\{r(v,W): v\in V(G)\}$ vsebuje n različnih metričnih predstavitev glede na množico W. Rešljiva množica grafa G z najmanjšim številom vozlišč se imenuje \textbf{metrična baza grafa G}. Število vozlišč metrične baze G pa je \textbf{metrična dimenzija grafa G}, ki jo označujemo z \textbf{$dim(G)$}. W je rešljiva množica, če ne obstaja podmnožica množice W, ki bi bila rešljiva množica.  

\section*{Načrt za reševanje}
Nalogo bova reševali s sledečim ILP programom.\\

Definiramo $V_p$ kot množico $\binom{n}{2}$ parov vozlišč grafa G. Za vsak par $u,v \in V(G)$ definiramo njuno $R\{u,v\}=\{x \in V(G): d(u,x)\neq d(v,x)\}.$ Tudi za vsak $x \in V(G)$ definiramo $R\{x\}= \{\{u,v\}\in V_p: d(u,x)\neq d(v,x)\}$. Z $R(G)$ označimo dvodelen graf, sestavljen iz dveh disjunktnih množic $V$ in $V_p$, kjer je $x \in V$ povezan s parom $\{u,v\} \in V_p$, če velja $d(u,x)\neq d(v,x)$ za graf G.  Tako je minimalna moč $S \subseteq V$ grafa $R(G)$, za katero velja $N(S) = V_p$  , metrična dimenzija grafa G. \\\\
Naj bo:\\
$V = \{v_1,\dots, v_n\}$ in $V_p = \{s_1,s_2,\dots,s_{\binom{n}{2}} \} $. Definiramo matriko  $ A = (a_{ij})$  velikosti $\binom{n}{2} \times n$, kjer je\[ a_{i,j}  =  \left \{ \begin{array}{l}1,\ \mbox{$ s_i v_j \in E(R(G))$}; \\0,\ \mbox{sicer}. \end{array} \right. \]
za $1\le i\le \binom{n}{2}$ in $1\le  j\le n$.\\
\pagebreak
\\
 Minimiziramo funkcijo: \begin{center}$f(x_1,x_2, \cdots,x_n)=x_1+x_2+\cdots+x_n$ 
ob upoštevanju $A\bar{x}\ge\ \bar{1}$,\end{center} 
kjer je: \begin{center}
 $\bar{x}=(x_1,\cdots,x_n)^T$,\\
$x_i\ \in\ \{0,1 \}$,\\
$\bar{1}$ je $\binom{n}{2}\times 1$  stolpični vektor samih enic. \end{center}
Če namesto binarnih spremenljivk $x_i$ zahtevamo realne spremenljivke, pa rešujemo problem \textbf{deljene metrične dimenzije}. \\\\
Obravnavali bova metrično in deljeno metrično dimenzijo na različnih oblikah grafov, kot so npr. drevesa in hiperkocke. Na koncu bova še določili metrično dimenzijo na kartezičnem produktu grafov. 

\end{document}