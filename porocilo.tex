\documentclass[a4paper]{report}
\usepackage[slovene]{babel}
\usepackage[utf8]{inputenc}
\usepackage[T1]{fontenc}
%\usepackage{bbm}
\usepackage{amsmath}
%\usepackage{blindtext}
\usepackage{array,multirow,graphicx}

\title{Metrična dimenzija grafa }
\author{Nastja Košir, Ajda Stare}
\date{\today}
\begin{document}
	\maketitle

\section*{Navodilo}
For an ordered subset $W = (w_1 , \dots, w_k ) $ of vertices of a graph G, the distance vector of a vertex v $\in$ V(G)  to  W  is  $(d(v, w_1 ), d(v, w_2 ), \dots, d(v, w_k ))$ . The metric dimension of a graph G is the size of the smallest set of vertices W  $\subseteq $ V(G) such that every two different vertices of the graph have distinct distance vectors. Solve the metric dimension problem as ILP. Its LP relaxation is known as the fractional metric dimension of a graph. Experimentally determine how much these two invariants can differ. Also, try to determine them for trees or any other class of your interest. How far you can go with determining the metric dimension of hypercubes, or some other cartesian product graphs?\\

\section*{Opis problema}
Naj bo $G=(V,E)$ neusmerjen, povezan graf brez zank in večkratnih vozlišč. Za urejeno podmnožico $ W =(w_1 , \dots, w_k )$ vozlišč grafa G, definiramo\\$r(v|W) = (d(v,w_1),\dots, d(v,w_k))$, kot  metrično predstavitev vozlišča $v$ , glede na množico W. Množico W imenujemo \textbf{rešljiva (razrešitvena) množica} grafa G, če za poljubni  različni vozlišči $u,v$ grafa G velja $r(u|W)\neq r(v|W)$ (če bi veljalo $r(u|W)= r(v|W)$ bi pomenilo $u = v$ ). Torej, če je W rešljiva množica moči k, za graf G reda n, potem množica $\{r(v,W): v\in V(G)\}$ vsebuje n različnih metričnih predstavitev glede na množico W. Rešljiva množica grafa G z najmanjšim številom vozlišč se imenuje \textbf{metrična baza grafa G}. Število vozlišč metrične baze G pa je \textbf{metrična dimenzija grafa G}, ki jo označujemo z \textbf{$dim(G)$}. W je rešljiva množica, če ne obstaja podmnožica množice W, ki bi bila rešljiva množica.  

\section*{Metrična in deljena metrična dimenzija za splošne grafe}

Nalogo bova reševali s sledečim ILP programom.\\

Definiramo $V_p$ kot množico $\binom{n}{2}$ parov vozlišč grafa G. Za vsak par $u,v \in V(G)$ definiramo njuno $R\{u,v\}=\{x \in V(G): d(u,x)\neq d(v,x)\}.$ Tudi za vsak $x \in V(G)$ definiramo $R\{x\}= \{\{u,v\}\in V_p: d(u,x)\neq d(v,x)\}$. Z $R(G)$ označimo dvodelen graf, sestavljen iz dveh disjunktnih množic $V$ in $V_p$, kjer je $x \in V$ povezan s parom $\{u,v\} \in V_p$, če velja $d(u,x)\neq d(v,x)$ za graf G.  Tako je minimalna moč $S \subseteq V$ grafa $R(G)$, za katero velja $N(S) = V_p$  , metrična dimenzija grafa G. \\\\
Naj bo:\\
$V = \{v_1,\dots, v_n\}$ in $V_p = \{s_1,s_2,\dots,s_{\binom{n}{2}} \} $. Definiramo matriko  $ A = (a_{ij})$  velikosti $\binom{n}{2} \times n$, kjer je\[ a_{i,j}  =  \left \{ \begin{array}{l}1,\ \mbox{$ s_i v_j \in E(R(G))$}; \\0,\ \mbox{sicer}. \end{array} \right. \]
za $1\le i\le \binom{n}{2}$ in $1\le  j\le n$.\\

 Minimiziramo funkcijo: \begin{center}$f(x_1,x_2, \cdots,x_n)=x_1+x_2+\cdots+x_n$ 
ob upoštevanju $A\bar{x}\ge\ \bar{1}$,\end{center} 
kjer je: \begin{center}
 $\bar{x}=(x_1,\cdots,x_n)^T$,\\
$x_i\ \in\ \{0,1 \}$,\\
$\bar{1}$ je $\binom{n}{2}\times 1$  stolpični vektor samih enic. \end{center}
Če namesto binarnih spremenljivk $x_i$ zahtevamo realne spremenljivke, pa rešujemo problem \textbf{deljene metrične dimenzije}. \\

Zgornja linearna programa sva nato uporabili na splošnih grafih, ki sva jih generirali s pomočjo Sage-ove funkcije nauty\_geng. Uporabili sva samo povezane grafe, saj sta obe metrični dimenziji definirani le zanje. Hitro sva ugotovili, da je največja razlika med metrično in deljeno metrično dimenzijo dosežena pri polnih grafih. Metrična dimenzija grafa $K_n$ je enaka $n-1$, deljena metrična dimenzija pa $n/2$. Tako njuna razlika znaša $n/2-1$. \\

Zanimalo naju je tudi, kako s številom vozlišč narašča časovna zahtevnost algoritma. Za grafe do vključno 8 vozlišč, sva izračunali povprečje časa izračuna na vseh povezanih grafih z določenim številom vozlišč. Za večje grafe je bilo to časovno prezahtevno, zato sva povprečili le preko nekaj naključno izbranih grafov na nekem številu vozlišč. Spodnja tabela prikazuje povprečen čas izvajanja za metrično in deljeno metrično dimenzijo glede na število vozlišč.

\begin{center}
\resizebox{13cm}{!} {
\begin{tabular}{||c| |*{8}{c|}|}
\hline\hline
Število vozlišč & 1 & 2 & 3 & 4 & 5 & 6 & 7 \\
\hline
Metrična dimenzija [ms] & 131,900 & 69,784 & 35,487 & 13,845 & 4,174 & 4,698 & 3,319 \\
\hline
Deljena metrična dimenzija [$\mu$s] & 1,907 & 2,861 & 1,431 & 3.179 & 1,022 & 2,767 & 2,516 \\
\hline\hline
\end{tabular}}
\end{center}

\section*{Metrična in deljena metrična dimenzija za drevesa}

Najin algoritem za računanje metrične dimenzije drevesa najprej upošteva dejstvo, da je metrična dimenzija poti enaka 1. Drevo je pot, če ima natanko dve vozlišči stopnje 1 oziroma 2 lista, vsa ostala vozlišča pa so stopnje 2. V primeru, ko drevo ni pot, si pomagamo z neenakostjo (iz vira \cite{1}) $$n \le \frac{( \beta D + 4)(D + 2)}{8}$$ oziroma $$\beta \ge \frac{4(2n - D - 2)}{D(D + 2)},$$ kjer je $n$ število vozlišč drevesa, $\beta$ metrična dimenzija grafa in $D$ premer grafa. Kadar je $D$ sodo, neenakost velja z enačajem. Tako algoritem za sode $D$ vrne metrično dimenzijo z eksplicitno izražavo, za lihe pa doda novo omejitev v celoštevilski linearni program za izračun metrične dimenzije pri splošnih grafih.\\

Podobno algoritem za računanje deljene metrične dimenzije najprej preveri ali je drevo morda pot in takrat vrne 1, sicer pa nadaljuje enako kot v programu za splošne grafe. \\ 



\begin{thebibliography}{99}

\bibitem{1} \textit{Metric dimension (graph theory)}, [ogled 19. 11. 2018], dostopno na %\url{https://en.wikipedia.org/wiki/Metric_dimension_(graph_theory)}.

\bibitem{2} S. Arumugam, V. Mathew, \textit{The fractional metric dimension of graphs}, Discrete Mathematics \textbf{312},  (2012), 1584–1590.

\end{thebibliography}

\end{document}